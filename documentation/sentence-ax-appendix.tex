\documentclass[12pt]{article}
\usepackage{graphicx}     %Note graphics<graphicx<epsfig
\usepackage[table,xcdraw, dvipsnames]{xcolor}
\usepackage{hyperref} %written by Sebastian Rahtz
%\usepackage{subeqn}    %allows equations 1a, 1b
%\usepackage{subfig}    %allows figures 1a, 1b
\usepackage{amsmath}
\usepackage{amssymb}
\usepackage{xurl}
\usepackage{bbm}%for indicator function
%\bibliographystyle{alpha}
\usepackage{floatrow}


\usepackage[toc,page]{appendix}
\usepackage[nottoc]{tocbibind}

\usepackage[color,matrix,frame,arrow,curve]{xy}
\usepackage[shortlabels]{enumitem}
\usepackage{array}
\usepackage{pdflscape}
\usepackage{multicol, multirow}
\usepackage{ctable}
\usepackage{dirtree}
\usepackage{subfig}
\usepackage{longtable}

%\usepackage{fdsymbol}
%\usepackage{animate}% play animation

\usepackage[vlined,ruled]{algorithm2e}
\newcommand{\mycommfont}[1]{\small\ttfamily\textcolor{blue}{#1}}
\SetCommentSty{mycommfont}

%highlighting
\usepackage{mdframed}

\usepackage[makeroom]{cancel}

\usepackage{pdfpages}
\usepackage{hyperxmp}

% this escapes the underscore in text
%\usepackage[T1]{fontenc}
%\usepackage{underscore}

\paperheight=11in
\topmargin=0in
\headheight=0in
\headsep=0in
\topskip=0in
\textheight=8.5in
\footskip=.5in


\paperwidth=8.5in
\oddsidemargin=.25in
\evensidemargin=.25in
\textwidth=6.0in
\parindent=.5in

\newcommand{\chapquote}[3]{\begin{quotation} \textit{#1} \end{quotation} \begin{flushright} - #2, \textit{#3}\end{flushright} }

\newtheorem{claim}{Claim}
\newcommand{\proof}[0]{{\bf proof:} }
\newcommand{\qed}[0]{\quad\newline \noindent{\bf QED }}


\newcommand{\bra}[1]{\langle#1|}
\newcommand{\ket}[1]{|#1\rangle}
\newcommand{\av}[1]{\left\langle#1\right\rangle}
\newcommand{\pder}[2]{\frac{\partial#1}{\partial#2}}
\newcommand{\der}[2]{\frac{d#1}{d#2}}
\newcommand{\tr}[0]{{\rm tr }}
\newcommand{\beq}{\begin{equation}}
\newcommand{\eeq}{\end{equation}}
\newcommand{\bsub}{\begin{subequations}}
	\newcommand{\esub}{\end{subequations}}
\newcommand{\beqa}{\begin{eqnarray}}
\newcommand{\eeqa}{\end{eqnarray}}
\newcommand{\rarrow}[0]{\rightarrow}
\newcommand{\larrow}[0]{\leftarrow}
\newcommand{\uarrow}[0]{\uparrow}
\newcommand{\darrow}[0]{\downarrow}
\newcommand{\Rarrow}[0]{\Rightarrow}
\newcommand{\nRarrow}[0]{\nRightarrow}
\newcommand{\Larrow}[0]{\Leftarrow}
\newcommand{\nLarrow}[0]{\nLeftarrow}
\newcommand{\ul}[1]{\underline{#1}}
\newcommand{\ol}[1]{\overline{#1}}
\newcommand{\WW}[0]{ {\mathbb{W}}}
\newcommand{\ZZ}[0]{ {\mathbb{Z}}}
\newcommand{\RR}[0]{{ \mathbb{R}} }
\newcommand{\CC}[0]{{ \mathbb{C}} }
\newcommand{\XX}[0]{{ \mathbb{X}} }
\newcommand{\ground}{{}_{\stackrel{\stackrel{\displaystyle{\bot}}{-}}{.}}}
\newcommand{\norm}[1]{\parallel#1\parallel}
\newcommand{\eqdef}[0]{\;\;\stackrel{\text{def}}{=}\;\;}

\newcommand{\heavy}[0]{\mathbf{H}}

\newcommand{\rva}[0]{{\ul{a}}}
\newcommand{\rvb}[0]{{\ul{b}}}
\newcommand{\rvc}[0]{{\ul{c}}}
\newcommand{\rvd}[0]{{\ul{d}}}
\newcommand{\rve}[0]{{\ul{e}}}
\newcommand{\rvf}[0]{{\ul{f}}}
\newcommand{\rvg}[0]{{\ul{g}}}
\newcommand{\rvh}[0]{{\ul{h}}}
\newcommand{\rvi}[0]{{\ul{i}}}
\newcommand{\rvj}[0]{{\ul{j}}}
\newcommand{\rvk}[0]{{\ul{k}}}
\newcommand{\rvl}[0]{{\ul{l}}}
\newcommand{\rvll}[0]{{\ul{\ell}}}
\newcommand{\rvm}[0]{{\ul{m}}}
\newcommand{\rvn}[0]{{\ul{n}}}
\newcommand{\rvo}[0]{{\ul{o}}}
\newcommand{\rvp}[0]{{\ul{p}}}
\newcommand{\rvq}[0]{{\ul{q}}}
\newcommand{\rvr}[0]{{\ul{r}}}
\newcommand{\rvs}[0]{{\ul{s}}}
\newcommand{\rvt}[0]{{\ul{t}}}
\newcommand{\rvu}[0]{{\ul{u}}}
\newcommand{\rvv}[0]{{\ul{v}}}
\newcommand{\rvw}[0]{{\ul{w}}}
\newcommand{\rvx}[0]{{\ul{x}}}
\newcommand{\rvy}[0]{{\ul{y}}}
\newcommand{\rvz}[0]{{\ul{z}}}


\newcommand{\rvA}[0]{{\ul{A}}}
\newcommand{\rvB}[0]{{\ul{B}}}
\newcommand{\rvC}[0]{{\ul{C}}}
\newcommand{\rvD}[0]{{\ul{D}}}
\newcommand{\rvE}[0]{{\ul{E}}}
\newcommand{\rvF}[0]{{\ul{F}}}
\newcommand{\rvH}[0]{{\ul{H}}}
\newcommand{\rvK}[0]{{\ul{K}}}
\newcommand{\rvL}[0]{{\ul{L}}}
\newcommand{\rvN}[0]{{\ul{N}}}
\newcommand{\rvQ}[0]{{\ul{Q}}}
\newcommand{\rvR}[0]{{\ul{R}}}
\newcommand{\rvT}[0]{{\ul{T}}}
\newcommand{\rvU}[0]{{\ul{U}}}
\newcommand{\rvV}[0]{{\ul{V}}}
\newcommand{\rvW}[0]{{\ul{W}}}
\newcommand{\rvX}[0]{{\ul{X}}}
\newcommand{\rvY}[0]{{\ul{Y}}}
\newcommand{\rvZ}[0]{{\ul{Z}}}

\newcommand{\rvxi}[0]{{\ul{\xi}}}
\newcommand{\rvzeta}[0]{{\ul{\zeta}}}
\newcommand{\rvtheta}[0]{{\ul{\theta}}}
\newcommand{\rvmu}[0]{{\ul{\mu}}}
\newcommand{\rvsig}[0]{{\ul{\sigma}}}
\newcommand{\rvDel}[0]{{\ul{\Delta}}}
\newcommand{\rvtau}[0]{{\ul{\tau}}}

\newcommand{\cala}[0]{{\cal A}}
\newcommand{\calb}[0]{{\cal B}}
\newcommand{\calc}[0]{{\cal C}}
\newcommand{\cald}[0]{{\cal D}}
\newcommand{\cale}[0]{{\cal E}}
\newcommand{\calf}[0]{{\cal F}}
\newcommand{\calg}[0]{{\cal G}}
\newcommand{\calh}[0]{{\cal H}}
\newcommand{\cali}[0]{{\cal I}}
\newcommand{\calk}[0]{{\cal K}}
\newcommand{\call}[0]{{\cal L}}
\newcommand{\calm}[0]{{\cal M}}
\newcommand{\caln}[0]{{\cal N}}
\newcommand{\calo}[0]{{\cal O}}
\newcommand{\calp}[0]{{\cal P}}
\newcommand{\calq}[0]{{\cal Q}}
\newcommand{\calr}[0]{{\cal R}}
\newcommand{\cals}[0]{{\cal S}}
\newcommand{\calt}[0]{{\cal T}}
\newcommand{\calu}[0]{{\cal U}}
\newcommand{\calv}[0]{{\cal V}}
\newcommand{\calw}[0]{{\cal W}}
\newcommand{\calx}[0]{{\cal X}}
\newcommand{\caly}[0]{{\cal Y}}
\newcommand{\calz}[0]{{\cal Z}}

\newcommand{\Pmat}[4]{\calp\left[
\begin{array}{cc}#1&#2\\#3&#4
\end{array}\right]}

%\newcommand{\PN}[0]{PN^{0,0}_{1,1}}
%\newcommand{\PS}[0]{PS^{1,1}_{0,0}}
\newcommand{\PN}[0]{PN}
\newcommand{\PS}[0]{PS}

\newcommand{\lam}[0]{\lambda}
\newcommand{\Lam}[0]{\Lambda}
\newcommand{\alp}[0]{\alpha}
\newcommand{\eps}[0]{\epsilon}
\newcommand{\s}[0]{\sigma}
\newcommand{\su}[0]{{\Sigma}}

\newcommand{\pp}[0]{\mathbb{P}}
\newcommand{\dbm}[0]{{
		[1,\partial_{b},\partial_{m}]
}}

\newcommand{\dg}[0]{{[1, \partial_{\theta_G}]}}
\newcommand{\dd}[0]{{[1, \partial_{\theta_D}]}}
\newcommand{\dgd}[0]{{[1, \partial_{\theta_G}, \partial_{\theta_D}]}}

\newcommand{\veca}[0]{{\vec{a}}}
\newcommand{\vecb}[0]{{\vec{b}}}
\newcommand{\vecc}[0]{{\vec{c}}}
\newcommand{\vecd}[0]{{\vec{d}}}
\newcommand{\vecf}[0]{{\vec{f}}}
\newcommand{\vech}[0]{{\vec{h}}}
\newcommand{\vecr}[0]{{\vec{r}}}
\newcommand{\vecs}[0]{{\vec{s}}}
\newcommand{\vecu}[0]{{\vec{u}}}
\newcommand{\vecx}[0]{{\vec{x}}}
\newcommand{\vecy}[0]{{\vec{y}}}
\newcommand{\vechy}[0]{{\vec{\haty}}}
\newcommand{\vtheta}[0]{{\vec{\theta}}}

\newcommand{\haty}[0]{{\widehat{y}}}
\newcommand{\hatx}[0]{{\widehat{x}}}
\newcommand{\hata}[0]{{\widehat{a}}}
\newcommand{\hatr}[0]{{\widehat{r}}}


\newcommand{\cond}[0]{{\:\mathbf{|}\:}}
\newcommand{\mymathbf}[1]{#1}

\newcommand{\ranvec}[1]{\ul{\vec{#1}}}
\newcommand{\indi}[0]{\mathbbm{1}}
\newcommand{\smoid}[0]{{\rm smoid}}
\newcommand{\lodds}[0]{{\rm lodds}}
\newcommand{\expit}[0]{{\rm expit}}
\newcommand{\logit}[0]{{\rm logit}}
\newcommand{\sign}[0]{{\rm sign}}

\renewcommand{\labelitemii}{$\bullet$}

\newcommand{\HAT}[1]{{\widehat{#1}}}
\newcommand{\TIL}[1]{{\widetilde{#1}}}
\newcommand{\tild}[0]{{\TIL{d}}}
\newcommand{\tile}[0]{{\TIL{e}}}
\newcommand{\tilg}[0]{{\TIL{g}}}
\newcommand{\tilu}[0]{{\TIL{u}}}
\newcommand{\tilx}[0]{{\TIL{x}}}
\newcommand{\tilP}[0]{{\TIL{P}}}
\newcommand{\tilPT}[0]{{\TIL{P}_\theta}}

\newcommand{\maparrow}[1]
{\xymatrix{\ar[r]_{#1}&}}


\newcommand{\ucalm}[0]{\ul{\calm}}

\newcommand{\bool}[0]{\{0,1\}}

\newcommand{\argmin}{\mathop{\mathrm{argmin}}\limits}
\newcommand{\argmax}{\mathop{\mathrm{argmax}}\limits}

\newcommand{\softmax}[0]{{\rm softmax}}

\newcommand{\A}[0]{\wedge}
\newcommand{\V}[0]{\vee}
\newcommand{\xor}{\oplus}
\newcommand{\bigA}[0]{\bigwedge}
\newcommand{\bigV}[0]{\bigvee}
\newcommand{\bigxor}{\bigoplus}


\newcommand{\rdart}[0]{\Rightarrow}
\newcommand{\ldart}[0]{\Leftarrow}
\newcommand{\rveps}[0]{\ul{\eps}}

\newcommand{\hatvar}[0]{\widehat{\sigma^2}}
\newcommand{\ptp}[0]{{(t)}}

\newcommand{\tseries}[1]{{\{#1\}_{\forall t}}}

\newcommand{\xbeta}[0]{X_\s^T\beta}
\newcommand{\xtau}[0]{X_\s^T\tau}

%arguments phi1, phi2, phi3, e
\newcommand{\rbd}[4]{
\xymatrix@-1.3pc{
&#1\ar[d]&&#2\ar[d]
\\
&\rvx_1\ar[rr]
&&
\rvx_2
\ar `r[rd][rd]
\\
0\ar`u[u][ru]
\ar`d[dr][rrd]
&&&&\rvA\ar[r]&#4
\\
&&
\rvx_3
\ar `r[rru][rru]
\\
&&#3\ar[u]
}
}

\newcommand{\rulezeroif}[0]{
If $(\rvb. \perp \rva.
|\rvr., \rvs.)$
in $G$, then}

\newcommand{\rulezerothen}[0]{
$\rva.=a. \leftrightarrow 1$}

\newcommand{\rulezeroP}[0]{
P(b.|a.,r.,s.)=P(b.|r., s.)}

\newcommand{\rulezeroH}[0]{
H(\rvb.:\rva.|\rvr.,\rvs.)=0}

\newcommand{\rulezeropic}[0]{
\xymatrix@C=1pc@R=1pc{
&r.\ar[dr]
&s.\ar[d]
\\
a.\ar[rr]
&&b.
&=
}
\xymatrix@C=1pc@R=1pc{
&r.\ar[dr]&s.\ar[d]
\\
a.\ar[rr]|0
&&b.
}}

\newcommand{\ruleoneif}[0]{
If $(\rvb. \perp \rva.
|\rvr., \rvs.)$
in $\cald_{\rvr.} G$, then}

\newcommand{\ruleonethen}[0]{
$\rva.=a. \leftrightarrow 1$}

\newcommand{\ruleoneP}[0]{
P(b.|a., \cald\rvr.=r.,s.)=
P(b.|\cald\rvr.=r., s.)}

\newcommand{\ruleoneH}[0]{
H(\rvb.:\rva.|\cald\rvr.,\rvs.)=0}

\newcommand{\ruleonepic}[0]{
\xymatrix@C=1pc@R=1pc{
&\cald\rvr.=r.\ar[dr]
&s.\ar[d]
\\
a.\ar[rr]
&&b.
&=
}
\xymatrix@C=1pc@R=1pc{
&\cald\rvr.=r.\ar[dr]
&s.\ar[d]
\\
a.\ar[rr]|0
&&b.
}}

\newcommand{\ruletwoif}[0]{
If $(\rvb.\perp \rva. |
 \rvr., \rvs.)$
in $\call_{\rva.}\cald_{\rvr.} G$,
 then}

\newcommand{\ruletwothen}[0]{
$\cald \rva.=a. \leftrightarrow \rva.=a.$}

\newcommand{\ruletwoP}[0]{
P(b.|\cald\rva.=a., \cald\rvr.=r., s.)=
P(b.|a., \cald\rvr.=r.,  s.)}

\newcommand{\ruletwoH}[0]{
H(\rvb.:\cald\rva.|\cald\rvr.,  \rvs.)
=
H(\rvb.:\rva.|\cald\rvr.,  \rvs.)}

\newcommand{\ruletwopicA}[0]{
\xymatrix@C=1pc@R=1pc{
\;\ar[d]_0
&\cald\rvr.=r.\ar[dr]
&s.\ar[d]
\\
\cald\rva.=a.\ar[rr]\ar[d]
&&b.
&=\quad\quad
\\
&
}
\xymatrix@C=1pc@R=1pc{
\;\ar[d]_0
&&\cald\rvr.=r.\ar[dr]
&s.\ar[d]
\\
\cald\rva.=a.\ar[d]
&a.\ar[rr]
&&b.
\\
&
}}
\newcommand{\ruletwopicB}[0]{
\xymatrix@C=1pc@R=1pc{
\;\ar[d]
&\cald\rvr.=r.\ar[dr]
&s.\ar[d]
\\
a.\ar[rr]\ar[d]
&&b.
&=\quad\quad
\\
&
}
\xymatrix@C=1pc@R=1pc{
\;\ar[d]
&&\cald\rvr.=r.\ar[dr]
&s.\ar[d]
\\
a.\ar[d]
&\cald\rva.=a.\ar[rr]
&&b.
\\
&
}}

\newcommand{\rulethreeif}[0]{
If $
(\rvb. \perp \rva.
| \rvr., \rvs.)$
in $\cald_{\rva.-an(\rvs.)}
\cald_{\rvr.}G$,
 then}

\newcommand{\rulethreethen}[0]{
$\cald \rva.=a. \leftrightarrow 1$}

\newcommand{\rulethreeP}[0]{
P(b.|\cald\rva.=a.,\cald\rvr.=r.,  s.)=
P(b.|\cald\rvr.=r., s.)}

\newcommand{\rulethreeH}[0]{
H(\rvb.:\cald\rva.|\cald\rvr., \rvs.)=0}

\newcommand{\rulethreepic}[0]{
\xymatrix@C=1pc@R=1pc{
&
&\cald\rvr.=r.\ar[dr]
&s.\ar[d]
\\
&\cald\rva.=a.\ar[rr]
&&b.
&=
}
\xymatrix@C=1pc@R=1pc{
&
&\cald\rvr.=r.\ar[dr]
&s.\ar[d]
\\
&\cald\rva.=a.\ar[rr]|0
&&b.
}}

\newcommand{\bdoordef}[0]{
Suppose that we have access to data
that allows us to
estimate a probability
distribution
 $P(x., y., z.)$.
Hence, the variables
$\rvx., \rvy., \rvz.$ are
ALL observed (i.e, not hidden).
Then we say that the
backdoor $\rvz.$
satisfies the
{\bf backdoor adjustment criterion}
relative to $(\rvx., \rvy.)$
if
\begin{enumerate}
\item
All backdoor paths from $\rvx.$
to $\rvy.$
 are blocked by  conditioning on
 $\rvz.$.
\item
$\rvz. \cap de(\rvx.)=\emptyset$.
\end{enumerate}
}

\newcommand{\bdoorclaim}[0]{
If $\rvz.$ satisfies the
backdoor criterion relative to
 $(\rvx., \rvy.)$, then

\beqa
P(y. | \cald \rvx.=x.)&=&
\sum_{z.} P(y.|x., z.)P(z.)
\\
&=&
\begin{array}{l}
\\
\\
\end{array}
\xymatrix{
\sum z.\ar[dr]
\\
x.\ar[r]
&y.
}
\eeqa
where $\sum z.$ means node
$\rvz.$ is summed over.
}

\newcommand{\fdoordef}[0]{
Suppose that we have access to data
that allows us to
estimate a probability
distribution
 $P(x., m., y.)$.
Hence, the variables
$\rvx., \rvm., \rvy.$ are
ALL observed (i.e, not hidden).
Then we say that
the frontdoor $\rvm.$
satisfies the
{\bf frontdoor adjustment criterion}
relative to $(\rvx., \rvy.)$
if
\begin{enumerate}
\item
All directed paths from
$\rvx.$ to $\rvy.$ are intercepted by
(i.e., have a node in) $\rvm.$.
\item
All backdoor paths from $\rvx.$ to
$\rvm.$ are blocked.
\item
All backdoor paths from
on $\rvm.$ to $\rvy.$
are blocked by conditioning
on  $\rvx.$.
\end{enumerate}
}

\newcommand{\fdoorclaim}[0]{
If $\rvm.$ satisfies the
frontdoor criterion
relative to $(\rvx., \rvy.)$,
and $P(x.,m.)>0$, then

\beqa
P(y. | \cald \rvx.=x.)&=&\sum_{m.}
\underbrace{\left[\sum_{x'.}
P(y.|x'., m.)P(x'.)\right]}_
{P(y.|\cald \rvm.=m.)}
\underbrace{P(m.|x.)}_
{P(m.|\cald \rvx.=x.)}
\\
&=&
\xymatrix{
&\sum x'.\ar[dr]
\\
x.´\ar[r]
&\sum m.\ar[r]&y.
}
\eeqa
where $\sum x'.$ and
$\sum m.$
means nodes
$\rvx'.$ and $\rvm.$
are summed over.
}


%Symmetry
\newcommand{\symrule}[0]{
$\rva\perp_P\rvb\implies \rvb\perp_P\rva$}

\newcommand{\symruleH}[0]{
$H(\rva:\rvb)=0\implies H(\rvb:\rva)=0$}

%Decomposition
\newcommand{\decrule}[0]{
$\rva\perp_P\rvb, \rvc\implies
\rva\perp_P\rvb \text{ and } \rva\perp_P\rvc$}

\newcommand{\decruleH}[0]{
$H(\rva:\rvb, \rvc)=0\implies
H(\rva:\rvb)=0 \text{ and } H(\rva:\rvc)=0$}

%Weak Union
\newcommand{\wearule}[0]{
$\rva\perp_P \rvb, \rvc \implies
\rva\perp_P\rvb|\rvc\text{ and }\rva\perp_P\rvc|\rvb$}

\newcommand{\wearuleH}[0]{
$H(\rva:\rvb, \rvc)=0 \implies
H(\rva:\rvb|\rvc)=0\text{ and }H(\rva:\rvc|\rvb)=0$}

%Contraction
\newcommand{\conrule}[0]{
$\rva\perp_P\rvb|\rvc\text{ and }\rva\perp_P \rvc
\implies \rva\perp_P \rvb, \rvc$}

\newcommand{\conruleH}[0]{
$H(\rva:\rvb|\rvc)=0\text{ and }H(\rva:\rvc)=0
\implies H(\rva:\rvb, \rvc)=0$}

%Intersection
\newcommand{\intrule}[0]{
$\rva\perp_P\rvb|\rvc, \rvd\text{ and }
\rva\perp_P \rvd|\rvc, \rvb\implies
\rva\perp_P \rvb,\rvd|\rvc$}

\newcommand{\intruleH}[0]{
$H(\rva:\rvb|\rvc, \rvd)=0\text{ and }
H(\rva:\rvd|\rvc, \rvb)=0\implies
H(\rva:\rvb,\rvd|\rvc)=0$}

\newcommand{\dotbarmu}[0]{{\cdot|\mu}}
\newcommand{\dotmu}[0]{{\cdot, \mu}}
\newcommand{\kbarmu}[0]{{k|\mu}}
\newcommand{\kmu}[0]{{k,\mu}}
\newcommand{\plusbarmu}[0]{{+|\mu}}
\newcommand{\plusmu}[0]{{+,\mu}}

\newcommand{\bnlearn}[0]{{\tt bnlearn\;}}

\newcommand{\sqsig}[0]{{[\sigma]}}

\newcommand{\misscellone}[0]{
\begin{array}{c}
\frac{1}{nsam}
P(x_0=0, x_2=0\cond x_1=1, \theta)
\\
\frac{1}{nsam}
P(x_0=0, x_2=1\cond x_1=1, \theta)
\\
\frac{1}{nsam}
P(x_0=1, x_2=0\cond x_1=1, \theta)
\\
\frac{1}{nsam}
P(x_0=1, x_2=1\cond x_1=1, \theta)
\end{array}
}

\newcommand{\misscelltwo}[0]{
\begin{array}{c}
\frac{1}{nsam}
P(x_1=0\cond x_0=0,x_2=1, \theta)
\\
\frac{1}{nsam}
P(x_1=1\cond x_0=0,x_2=1,  \theta)
\end{array}
}


\newcommand{\td}[0]{{\TIL{d}}}
\newcommand{\rvtd}[0]{{\ul{\TIL{d}}}}
\newcommand{\tx}[0]{{\TIL{x}}}
\newcommand{\tmu}[0]{{\TIL{\mu}}}
\newcommand{\rvtx}[0]{{\ul{\TIL{x}}}}

\newcommand{\mlarr}[0]{\xrightarrow{\rm ML-fit}}
\newcommand{\lrarr}[0]{\xrightarrow{\rm LR-fit}}

\newcommand{\setprob}[3]
{{\begin{array}{c}S=\{#1\}
\\P(S)=#2\\ \haty(x^\s_S)=\$#3 K
\end{array}}}

\newcommand{\Gno}[0]{\xymatrix{\;\ar[r]|\parallel_G&}}
\newcommand{\Gyes}[0]{\xymatrix{\;\ar[r]_G&}}

\newcommand{\calypso}[0]{\ol{\caly}}

\newcommand{\SeqBdoorDef}[0]{
Suppose that we have access to data
that allows us to
estimate a probability
distribution
 $P(x^n, y, z^n)$.
Hence, the variables
$\rvx^n, \rvy, \rvz^n$ are
ALL observed (i.e, not hidden).
Then we say that the
the multinode
of ``covariates" $\rvz^n$
satisfies the
{\bf sequential backdoor (SBD) adjustment criterion}
relative to $(\rvx^n, \rvy)$
if for all $t\in\{0,1, \ldots, n-1\}$,

\begin{enumerate}
\item
$\rvy\perp\rvx_t|
\underbrace{(\rvx_0, \rvx_1, \ldots,\rvx_{t-1},
\rvz_0, \rvz_1, \ldots, \rvz_t)}
_{\text{Past of $\rvx_t$}}$
in $\call_{\rvx_{t}}
\cald_{\rvx_{t+1},\rvx_{t+2}
,\ldots, \rvx_{n-1}}G$.
\item
$\rvz_t \cap de(\rvx_t)=\emptyset$.
\end{enumerate}
}

\newcommand{\SeqBdoorClaim}[0]{
If $\rvz^n$ satisfies the
sequential backdoor criterion relative to
 $(\rvx^n, \rvy)$, then

\beq
P(y | \cald \rvx^n=x^n)=
\calq(y|x^n)
\;,
\eeq
where $\calq(y|x^n)$
is defined by
Eq.(\ref{def-q-y-xn-seqbdoor}).
}




\begin{document}



\title{Appendix
to Bayesuvius Chapter about SentenceAx}
\date{ \today}
\author{Robert R. Tucci\\
        tucci@ar-tiste.com}
\maketitle
The SentenceAx (Sax) software (at github repo Ref.\cite{sentence-ax-github}) is a complete re-write of the Openie6 (O6) software
(at github repo Ref.\cite{openie6-github}).

 The O6 software is described by its creators
 in the paper Ref.\cite{openie6-paper},
 which we will henceforth refer to as 
 the O6 paper.
 
 Before reading this appendix, you should read 
 the document entitled ``Sentence Splitting with SentenceAx" (Ref.\cite{openie6-paper}) that is a
 chapter 
 excerpt from my book Bayesuvius (Ref.\cite{bayesuvius}). I  will henceforth refer to that chapter as
 the Sax chapter. The purpose of this Appendix is to record details about Sax that were deemed too fine-grained or ephemeral to be included in the Sax chapter.


\section{PyTorch code for calculating Penalty Loss}

The Sax chapter gives all the equations
associated with Penalty Loss. But how does one   code them with PyTorch?  The O6 software does it masterfully. Here is the pertinent code snippet from Sax.
It comes directly from the O6 software, modulus changes in notation.




\begin{lstlisting}[language=Python]
@staticmethod
def sax_penalty_loss(x_d,
                   llll_word_scoreT,
                   con_to_weight):
    """
    similar to Openie6.model.constrained_loss()

    This method is called inside sax_batch_loss(). It returns the
    penalty loss.

    Parameters
    ----------
    x_d: OrderedDict
    llll_word_scoreT: torch.Tensor
    con_to_weight: dict[str, float]

    Returns
    -------
    float
        penalty_loss

    """
    batch_size, num_depths, num_words, icode_dim = \
        llll_word_scoreT.shape
    penalty_loss = 0
    llll_index = x_d["ll_osent_verb_loc"].\
        unsqueeze(1).unsqueeze(3).repeat(1, num_depths, 1, icode_dim)
    llll_verb_trust = torch.gather(
        input=llll_word_scoreT,
        dim=2,
        index=llll_index)
    lll_verb_rel_trust = llll_verb_trust[:, :, :, 2]
    # (batch_size, depth, num_words)
    lll_bool = (x_d["ll_osent_verb_loc"] != 0).unsqueeze(1).float()

    lll_verb_rel_trust = lll_verb_rel_trust * lll_bool
    # every head-verb must be included in a relation
    if 'hvc' in con_to_weight:
        ll_column_loss = \
            torch.abs(1 - torch.sum(lll_verb_rel_trust, dim=1))
        ll_column_loss = \
            ll_column_loss[x_d["ll_osent_verb_loc"] != 0]
        penalty_loss += con_to_weight['hvc'] * ll_column_loss.sum()

    # extractions must have at least k-relations with 
    # a head verb in them
    if 'hvr' in con_to_weight:
        l_a = x_d["ll_osent_verb_bool"].sum(dim=1).float()
        l_b = torch.max(lll_verb_rel_trust, dim=2)[0].sum(dim=1)
        row_rel_loss = F.relu(l_a - l_b)
        penalty_loss += con_to_weight['hvr'] * row_rel_loss.sum()

    # one relation cannot contain more than one head verb
    if 'hve' in con_to_weight:
        ll_ex_loss = \
            F.relu(torch.sum(lll_verb_rel_trust, dim=2) - 1)
        penalty_loss += con_to_weight['hve'] * ll_ex_loss.sum()

    if 'posm' in con_to_weight:
        llll_index = \
            x_d["ll_osent_pos_loc"].unsqueeze(1).unsqueeze(3).\
            repeat(1, num_depths, 1, icode_dim)
        llll_pred_trust = torch.gather(
            input=llll_word_scoreT,
            dim=2,
            index=llll_index)
        lll_pos_not_none_trust = \
            torch.max(llll_pred_trust[:, :, :, 1:], dim=-1)[0]
        ll_column_loss = \
            (1 - torch.max(lll_pos_not_none_trust, dim=1)[0]) * \
            (x_d["ll_osent_pos_loc"] != 0).float()
        penalty_loss += con_to_weight['posm'] * ll_column_loss.sum()

    return penalty_loss
    
\end{lstlisting}



\section{Sax bnet}

The Sax chapter gives 
a drawing
of the Sax bnet,
and a list
of its structural
equations.
Both were produced with the texnn tool (Ref.\cite{texnn})

In this section,
we provide 
evidence that
 Sax
 does indeed implement
 that bnet correctly.

This section has 3 parts.

\begin{enumerate}

\item texnn output

\item Sax code
that implements the bnet.
\item Excerpt of print-out to console produced when I run
the jupyter notebook for training 
the warmup NN for task=ex. (The jupyter notebooks for warmup training have verbose=True. Those for non-warmup training have verbose=False).


\end{enumerate}

\subsection{texnn output}

\begin{figure}[h!]\centering
$$\xymatrix@R=2.5pc@C=3.5pc{
&&
\\
&&
\\
&*+[F*:Dandelion]{\underline{M}^{[86], [300]}}\ar[uu]\ar[r]^{W_{il}}&*+[F*:SkyBlue]{\underline{L}^{[86], [6]}}\ar[uu]
\\
*+[F*:pink]{\underline{S}^{[86], [768]}}\ar[ur]^{W_{me}}&*+[F*:pink]{\underline{E}^{[86], [768]}}\ar[l]^{1}&*+[F*:yellow]{\underline{a}^{[86]}}\ar[l]
\\
*+[F*:Orchid]{\underline{d}^{[121], [768]}}\ar[r]&*+[F*:pink]{\underline{G}^{[86], [768]}}\ar@[red]@/_5pc/[uu]|-{\color{red} W_{me}}\ar[ul]^{1}&
\\
&*+[F*:Orchid]{\underline{I}^{[121], [768]}}\ar[ul]&*+[F*:SkyBlue]{\underline{X}^{[86], [6]}}\ar[uu]
\\
&*+[F*:Orchid]{\underline{B}^{[121], [768]}}\ar[u]&
\save
\POS"3,1"."6,1"."3,3"."6,3"!C*+<4.8em>\frm{-,}
\POS"6,2"."6,2"."6,2"."6,2"!C*+<3.8em>\frm{--}
\POS"4,1"."4,1"."4,3"."4,3"!C*+<1.0em>\frm{.}
\restore
}$$
\caption{Sax bnet. 2 copies of dashed box are connected in series. 5 copies (5 depths) of plain box are connected in series.  However, in the first of those 5 plain box copies, the dotted box  is omitted and node $\ul{G}$ feeds directly into node  $\ul{M}$ (indicated by red arrow). We display the tensor shape superscripts in the PyTorch L2R order. All tensor shape superscripts have been simplified by omitting a $[s_{ba}]$ from their left side, where $s_{ba}=24$ is the batch size. }
\label{fig-texnn-for-sentence-ax-bnet}
\end{figure}

\begin{tabular}{ll}
$\underline{a}^{[86]}$ :&{\tt ll\_greedy\_ilabel}\\
$\underline{B}^{[121], [768]}$ :&{\tt lll\_hidstate}\\
$\underline{d}^{[121], [768]}$ :&{\tt lll\_hidstate}\\
$\underline{E}^{[86], [768]}$ :&{\tt lll\_pred\_code}\\
$\underline{G}^{[86], [768]}$ :&{\tt lll\_word\_hidstate}\\
$\underline{I}^{[121], [768]}$ :&{\tt lll\_hidstate}\\
$\underline{L}^{[86], [6]}$ :&{\tt lll\_word\_score}\\
$\underline{M}^{[86], [300]}$ :&{\tt lll\_word\_hidstate}\\
$\underline{S}^{[86], [768]}$ :&{\tt lll\_word\_hidstate}\\
$\underline{X}^{[86], [6]}$ :&{\tt lll\_word\_score}
\end{tabular}



\begin{subequations}

\begin{equation}\color{blue}
\begin{aligned}
a^{[86]} &= \text{argmax}(X^{[86], [6]};dim=-1)
\label{eq-a-fun-sentence-ax-bnet}
\\ &:{\tt ll\_greedy\_ilabel}
\end{aligned}
\end{equation}

\begin{equation}\color{blue}
\begin{aligned}
B^{[121], [768]} &= \text{BERT}()
\label{eq-B-fun-sentence-ax-bnet}
\\ &:{\tt lll\_hidstate}
\end{aligned}
\end{equation}

\begin{equation}\color{blue}
\begin{aligned}
d^{[121], [768]} &= \text{dropout}(I^{[121], [768]})
\label{eq-d-fun-sentence-ax-bnet}
\\ &:{\tt lll\_hidstate}
\end{aligned}
\end{equation}

\begin{equation}\color{blue}
\begin{aligned}
E^{[86], [768]} &= \text{embedding}(a^{[86]})
\label{eq-E-fun-sentence-ax-bnet}
\\ &:{\tt lll\_pred\_code}
\end{aligned}
\end{equation}

\begin{equation}\color{blue}
\begin{aligned}
G^{[86], [768]} &= \text{gather}(d^{[121], [768]};dim=-2)
\label{eq-G-fun-sentence-ax-bnet}
\\ &:{\tt lll\_word\_hidstate}
\end{aligned}
\end{equation}

\begin{equation}\color{blue}
\begin{aligned}
I^{[121], [768]} &= \left[B^{[121], [768]}\indi(depth=0)+ M^{[86], [300]}\indi(depth> 0)\right]
\label{eq-I-fun-sentence-ax-bnet}
\\ &:{\tt lll\_hidstate}
\end{aligned}
\end{equation}

\begin{equation}\color{blue}
\begin{aligned}
L^{[86], [6]} &= M^{[86], [300]}W_{il}^{[300],[6]}
\label{eq-L-fun-sentence-ax-bnet}
\\ &:{\tt lll\_word\_score}
\end{aligned}
\end{equation}

\begin{equation}\color{blue}
\begin{aligned}
M^{[86], [300]} &= \left[G^{[86], [768]}\indi(depth=0) + S^{[86], [768]}\indi(depth> 0) \right] W_{me}^{[768], [300]}
\label{eq-M-fun-sentence-ax-bnet}
\\ &:{\tt lll\_word\_hidstate}
\end{aligned}
\end{equation}

\begin{equation}\color{blue}
\begin{aligned}
S^{[86], [768]} &= E^{[86], [768]} + G^{[86], [768]}
\label{eq-S-fun-sentence-ax-bnet}
\\ &:{\tt lll\_word\_hidstate}
\end{aligned}
\end{equation}

\begin{equation}\color{blue}
\begin{aligned}
X^{[86], [6]} &= L^{[86], [6]}\indi(depth> 0)
\label{eq-X-fun-sentence-ax-bnet}
\\ &:{\tt lll\_word\_score}
\end{aligned}
\end{equation}

\end{subequations}


\subsection{ 
Sax code}

\begin{lstlisting}[language=Python]
def sax_get_llll_word_score(self, x_d, ttt, verbose=False):
"""

This method is used inside self.forward() and is the heart of that
method. It contains a while loop over depths that drives a batch
through the layers of the model and returns `llll_word_score`.
Setting `verbose` to True prints out a detailed trail of what occurs
in this method. The following example was obtained from such a
verbose trail.

Assume:
batch_size= 24,
hidden_size= 768,
NUM_ILABELS= 6,
MERGE_DIM= 300
2 iterative layers and 5 depths.

lll_word_score is the output of the last ilabelling_layer for each
depth

llll_word_score is a list of lll_word_score

len(llll_word_score)= 5 = num_depths

Note that llll_word_scoreT = Ten(llll_word_score)

Parameters
----------
x_d: OrderedDict
ttt: str
verbose: bool

Returns
-------
list[torch.Tensor]
    llll_word_score

"""
# lll_label is similar to Openie6.labels
# first (outer) list over batch/sample of events
# second list over extractions
# third (inner) list over number of labels in a line
# after padding and adding the 3 unused tokens

# batch_size, num_depths, num_words = y_d["lll_ilabel"].shape
# sometimes num_depths will exceed max.
# This doesn't happen when training, because
# num_depths is specified when training.
num_depths = get_num_depths(self.params.task)

# `loss_fun` is not used in this function anymore
# loss_fun, lstm_loss = 0, 0

# batch_text = " ".join(redoL(meta_d["l_orig_sent"]))
# prior_model_input = \
#     torch.Tensor(self.auto_tokenizer.encode(batch_text))
hstate_count = Counter(verbose, "lll_hidstate")
word_hstate_count = Counter(verbose, "lll_word_hidstate")
lll_hidstate, _ = self.prior_model(x_d["ll_osent_icode"])
hstate_count.new_one(reset=True)
comment(
    verbose,
    prefix="after prior_model",
    params_d={
        "ll_osent_icode.shape": x_d["ll_osent_icode"].shape,
        "lll_hidstate.shape": lll_hidstate.shape})
lll_word_score = Ten([0])  # this statement is unnecessary
llll_word_score = []  # ~ Openie6.all_depth_scores
depth = 0
# loop over depths
while True:
    for ilay, layer in enumerate(self.iterative_transformer):
        comment(verbose,
                prefix="*********** Starting iterative layer",
                params_d={"ilay": ilay})
        # layer(lll_hidstate)[0] returns a copy
        # of the tensor lll_hidstate after transforming it
        # in some way. [0] chooses first component
        comment(
            verbose,
            prefix="Before iterative layer",
            params_d={
                "ilay": ilay,
                "depth": depth,
                "lll_hidstate.shape": lll_hidstate.shape})
        lll_hidstate = layer(lll_hidstate)[0]
        hstate_count.new_one()
        comment(
            verbose,
            prefix="After iterative layer",
            params_d={
                "ilay": ilay,
                "depth": depth,
                "lll_hidstate.shape": lll_hidstate.shape})
    comment(verbose,
            prefix="Before dropout",
            params_d={
                "depth": depth,
                "lll_hidstate.shape": lll_hidstate.shape})
    lll_hidstate = self.dropout_fun(lll_hidstate)
    hstate_count.new_one()
    comment(verbose,
            prefix="After dropout",
            params_d={
                "depth": depth,
                "lll_hidstate.shape": lll_hidstate.shape})
    lll_loc = x_d["ll_osent_wstart_loc"].unsqueeze(2). \
        repeat(1, 1, lll_hidstate.shape[2])
    lll_word_hidstate = torch.gather(
        input=lll_hidstate,
        dim=1,
        index=lll_loc)
    comment(
        verbose,
        prefix="Gather's 2 inputs, then output",
        params_d={
            "lll_hidstate.shape": lll_hidstate.shape,
            "lll_loc.shape": lll_loc.shape,
            "lll_word_hidstate.shape": lll_word_hidstate.shape})
    word_hstate_count.new_one(reset=True)
    if depth != 0:
        comment(
            verbose,
            prefix="before argmax",
            params_d={"lll_word_score.shape": lll_word_score.shape})
        ll_greedy_ilabel = torch.argmax(lll_word_score, dim=-1)
        comment(
            verbose,
            prefix="after argmax",
            params_d={"ll_greedy_ilabel.shape":
                          ll_greedy_ilabel.shape})
        # not an integer code/embedding
        comment(
            verbose,
            prefix="before embedding",
            params_d={"ll_greedy_ilabel.shape":
                          ll_greedy_ilabel.shape})
        lll_pred_code = self.embedding(ll_greedy_ilabel)
        comment(
            verbose,
            prefix="after embedding",
            params_d={"lll_word_hidstate.state":
                          lll_word_hidstate.shape})
        lll_word_hidstate += lll_pred_code
        word_hstate_count.new_one()
        comment(
            verbose,
            prefix="just summed two signals with this shape",
            params_d={
                "depth": depth,
                "lll_word_hidstate.shape": lll_word_hidstate.shape})
    comment(verbose,
            prefix="Before merge layer",
            params_d={
                "depth": depth,
                "lll_word_hidstate.shape": lll_word_hidstate.shape})
    lll_word_hidstate = self.merge_layer(lll_word_hidstate)
    comment(
        verbose,
        prefix="After merge layer",
        params_d={
            "depth": depth,
            "lll_word_hidstate.shape": lll_word_hidstate.shape})
    comment(
        verbose,
        prefix="Before ilabelling",
        params_d={
            "depth": depth,
            "lll_word_hidstate.shape": lll_word_hidstate.shape})
    lll_word_score = self.ilabelling_layer(lll_word_hidstate)
    comment(
        verbose,
        prefix="After ilabelling",
        params_d={
            "depth": depth,
            "lll_word_score.shape": lll_word_score.shape})
    llll_word_score.append(lll_word_score)

    depth += 1
    if depth >= num_depths:
        break

    if ttt != 'train':
        ll_pred_ilabel = torch.max(lll_word_score, dim=2)[1]
        valid_extraction = False
        for l_pred_ilabel in ll_pred_ilabel:
            if is_valid_label_list(
                    l_pred_ilabel, self.params.task, "ilabels"):
                valid_extraction = True
                break
        if not valid_extraction:
            break
comment(
    verbose,
    params_d={
        "len(llll_word_score)": len(llll_word_score),
        "llll_word_score[0].shape": llll_word_score[0].shape})
return llll_word_score
\end{lstlisting}


\subsection{jupyter notebook print-out}


\begin{lstlisting}[language=Python]
"""
Entering Model.training_step method, batch_idx=0
'lll_hidstate' count changed: 0->1
after prior_model
    ll_osent_icode.shape=torch.Size([4, 121])
    lll_hidstate.shape=torch.Size([4, 121, 768])
*********** Starting iterative layer
    ilay=0
Before iterative layer
    ilay=0
    depth=0
    lll_hidstate.shape=torch.Size([4, 121, 768])
'lll_hidstate' count changed: 1->2
After iterative layer
    ilay=0
    depth=0
    lll_hidstate.shape=torch.Size([4, 121, 768])
*********** Starting iterative layer
    ilay=1
Before iterative layer
    ilay=1
    depth=0
    lll_hidstate.shape=torch.Size([4, 121, 768])
'lll_hidstate' count changed: 2->3
After iterative layer
    ilay=1
    depth=0
    lll_hidstate.shape=torch.Size([4, 121, 768])
Before dropout
    depth=0
    lll_hidstate.shape=torch.Size([4, 121, 768])
'lll_hidstate' count changed: 3->4
After dropout
    depth=0
    lll_hidstate.shape=torch.Size([4, 121, 768])
Gather's 2 inputs, then output
    lll_hidstate.shape=torch.Size([4, 121, 768])
    lll_loc.shape=torch.Size([4, 86, 768])
    lll_word_hidstate.shape=torch.Size([4, 86, 768])
'lll_word_hidstate' count changed: 0->1
Before merge layer
    depth=0
    lll_word_hidstate.shape=torch.Size([4, 86, 768])
After merge layer
    depth=0
    lll_word_hidstate.shape=torch.Size([4, 86, 300])
Before ilabelling
    depth=0
    lll_word_hidstate.shape=torch.Size([4, 86, 300])
After ilabelling
    depth=0
    lll_word_score.shape=torch.Size([4, 86, 6])
*********** Starting iterative layer
    ilay=0
Before iterative layer
    ilay=0
    depth=1
    lll_hidstate.shape=torch.Size([4, 121, 768])
'lll_hidstate' count changed: 4->5
After iterative layer
    ilay=0
    depth=1
    lll_hidstate.shape=torch.Size([4, 121, 768])
*********** Starting iterative layer
    ilay=1
Before iterative layer
    ilay=1
    depth=1
    lll_hidstate.shape=torch.Size([4, 121, 768])
'lll_hidstate' count changed: 5->6
After iterative layer
    ilay=1
    depth=1
    lll_hidstate.shape=torch.Size([4, 121, 768])
Before dropout
    depth=1
    lll_hidstate.shape=torch.Size([4, 121, 768])
'lll_hidstate' count changed: 6->7
After dropout
    depth=1
    lll_hidstate.shape=torch.Size([4, 121, 768])
Gather's 2 inputs, then output
    lll_hidstate.shape=torch.Size([4, 121, 768])
    lll_loc.shape=torch.Size([4, 86, 768])
    lll_word_hidstate.shape=torch.Size([4, 86, 768])
'lll_word_hidstate' count changed: 0->1
before argmax
    lll_word_score.shape=torch.Size([4, 86, 6])
after argmax
    ll_greedy_ilabel.shape=torch.Size([4, 86])
before embedding
    ll_greedy_ilabel.shape=torch.Size([4, 86])
after embedding
    lll_word_hidstate.state=torch.Size([4, 86, 768])
'lll_word_hidstate' count changed: 1->2
just summed two signals with this shape
    depth=1
    lll_word_hidstate.shape=torch.Size([4, 86, 768])
Before merge layer
    depth=1
    lll_word_hidstate.shape=torch.Size([4, 86, 768])
After merge layer
    depth=1
    lll_word_hidstate.shape=torch.Size([4, 86, 300])
Before ilabelling
    depth=1
    lll_word_hidstate.shape=torch.Size([4, 86, 300])
After ilabelling
    depth=1
    lll_word_score.shape=torch.Size([4, 86, 6])
"""
\end{lstlisting}






\bibliographystyle{plain}
\bibliography{references}
\end{document}